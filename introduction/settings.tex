
\mode<presentation> {

% The Beamer class comes with a number of default slide themes
% which change the colors and layouts of slides. Below this is a list
% of all the themes, uncomment each in turn to see what they look like.

%\usetheme{default}
%\usetheme{AnnArbor}
%\usetheme{Antibes}
%\usetheme{Bergen}
%\usetheme{Berkeley}
%\usetheme{Berlin}
\usetheme{Boadilla}
%\usetheme{CambridgeUS}
%\usetheme{Copenhagen}
%\usetheme{Darmstadt}
%\usetheme{Dresden}
%\usetheme{Frankfurt}
%\usetheme{Goettingen}
%\usetheme{Hannover}
%\usetheme{Ilmenau}
%\usetheme{JuanLesPins}
%\usetheme{Luebeck}
%\usetheme{Madrid}
%\usetheme{Malmoe}
%\usetheme{Marburg}
%\usetheme{Montpellier}
%\usetheme{PaloAlto}
%\usetheme{Pittsburgh}
%\usetheme{Rochester}
%\usetheme{Singapore}
%\usetheme{Szeged}
%\usetheme{Warsaw}

% As well as themes, the Beamer class has a number of color themes
% for any slide theme. Uncomment each of these in turn to see how it
% changes the colors of your current slide theme.

%\usecolortheme{albatross}
%\usecolortheme{beaver}
%\usecolortheme{beetle}
%\usecolortheme{crane}
%\usecolortheme{dolphin}
%\usecolortheme{dove}
%\usecolortheme{fly}
%\usecolortheme{lily}
%\usecolortheme{orchid}
%\usecolortheme{rose}
%\usecolortheme{seagull}
%\usecolortheme{seahorse}
%\usecolortheme{whale}
%\usecolortheme{wolverine}

%\setbeamercovered{transparent}
\setbeamercovered{invisible}

%\setbeamertemplate{footline} % To remove the footer line in all slides uncomment this line
%\setbeamertemplate{footline}[page number] % To replace the footer line in all slides with a simple slide count uncomment this line

\setbeamertemplate{navigation symbols}{} % To remove the navigation symbols from the bottom of all slides uncomment this line
}
% \usepackage[german]{babel}
\usepackage[utf8]{inputenc}
\usepackage{textcomp} % provides text symbols (e.g. euro symbol)
% \usepackage{pgfpages} % Fuer Unterstuetzung von mehreren Monitoren
% \setbeameroption{show notes on second screen} %zeigt Notizen auf dem 2. Monitor an

% \hypersetup{pdfpagemode=FullScreen} %AcrobatReader startet mit Vollbild

% \usepackage{multirow} % Um Zellen in Tabellen ueber mehrere Zeilen zu verbinden
% \usepackage{booktabs} % Fuer huebschere Tabellen
% \usepackage{wasysym} % einige Symbole
\usepackage{color}
\usepackage{colortbl} % for colored tables
\usepackage{siunitx}
\usepackage{lmodern}
% \usepackage{times}
\usepackage{amsmath}
\usepackage[T1]{fontenc}

\usepackage{listings} % fuer quellcode (C++, Python, ..)

% Oder was auch immer. Zu beachten ist, das Font und Encoding passen
% muessen. Falls T1 nicht funktioniert, kann man versuchen, die Zeile
% mit fontenc zu loeschen.

\usepackage{hyperref}

\usepackage{relsize} % for relative font size changes

\usepackage{booktabs} % Allows the use of \toprule, \midrule and \bottomrule in tables

\usepackage{tikz}
\usetikzlibrary{positioning,fit,decorations.pathmorphing,shapes.arrows,shapes.geometric,shapes.misc,shapes.multipart,calc,shadows,fadings}

\setlength{\unitlength}{\textwidth}  % measure in textwidths


% Falls eine Logodatei namens "university-logo-filename.xxx" vorhanden
% ist, wobei xxx ein von latex bzw. pdflatex lesbares Graphikformat
% ist, so kann man wie folgt ein Logo einfuegen:

% \pgfdeclareimage[height=0.5cm]{university-logo}{university-logo-filename}
% \logo{\pgfuseimage{university-logo}}

\frenchspacing

\newcommand{\unit}[1]{\, \mathrm{#1}}               % Einheiten in
% math. Formeln richtig darstellen

\newcommand{\entspricht}{\mathrel{\widehat{=}}}     % das "entspricht"-Symbol

\def\cel{^{\circ}C}                                  % Grad Celsius
\def\grad{^\circ}

\definecolor{grey}{rgb}{0.5,0.5,0.5}
\definecolor{aidablue}{RGB}{42,42,127}
\definecolor{olive}{rgb}{0.3, 0.4, .1}
\definecolor{fore}{RGB}{249,242,215}
\definecolor{back}{RGB}{51,51,51}
\definecolor{title}{RGB}{255,0,90}
\definecolor{dgreen}{rgb}{0.,0.6,0.}
\definecolor{gold}{rgb}{1.,0.84,0.}
\definecolor{JungleGreen}{cmyk}{0.99,0,0.52,0}
\definecolor{BlueGreen}{cmyk}{0.85,0,0.33,0}
\definecolor{RawSienna}{cmyk}{0,0.72,1,0.45}
\definecolor{Magenta}{cmyk}{0,1,0,0}

\definecolor{orangered}{rgb}{ 0.8441, 0.1582, 0.0000} % Orange-Red
\definecolor{lightorangered}{rgb}{ 0.75, 0.55, 0.0000} % Orange-Red

\definecolor{darkolive}{rgb}{0.667,0.604,0.016}
\definecolor{middleolive}{rgb}{0.859,0.776,0.020}
\definecolor{lightolive}{rgb}{0.980,0.894,0.094}

\definecolor{deepgreen}{rgb}{0.,0.5,0.}
\definecolor{darkgreen}{rgb}{0.078,0.667,0.016}
\definecolor{middlegreen}{rgb}{102,0.859,0.020}
\definecolor{lightgreen}{rgb}{180,0.980,0.094}

\definecolor{darkross}{rgb}{0.016,0.404,0.667}   % blue of Ross Seal
\definecolor{middleross}{rgb}{0.020,0.522,0.859}
\definecolor{lightross}{rgb}{0.094,0.624,0.980}

\definecolor{darkblue}{rgb}{0.016,0.078,0.667}
\definecolor{middleblue}{rgb}{0.020,0.102,0.856}
\definecolor{lightblue}{rgb}{0.094,0.180,0.980}

\definecolor{darkpurple}{rgb}{0.604,0.016,0.667}
\definecolor{middlepurple}{rgb}{0.776,0.020,0.895}
\definecolor{lightpurple}{rgb}{0.894,0.094,0.980}

\definecolor{Brown}{cmyk}{0, 0.8, 1, 0.6}
\definecolor{darkbrown}{rgb}{0.961,0.514,0.071} %bronze
\definecolor{middlebrown}{rgb}{0.725,0.557,0.271}
\definecolor{lightbrown}{rgb}{0.933,0.890,0.773} % beige of Ross-Seal

\definecolor{gold}{rgb}{0.85,0.65,0}

\newcommand{\blue}[1]{\textcolor{blue}{#1}}
\newcommand{\red}[1]{\textcolor{red}{#1}}
\newcommand{\olive}[1]{\textcolor{darkolive}{#1}}
\newcommand{\brown}[1]{\textcolor{brown!70!black}{#1}}
\newcommand{\gold}[1]{\textcolor{gold!70!black}{#1}}
\newcommand{\purple}[1]{\textcolor{middlepurple}{#1}}
\newcommand{\orange}[1]{\textcolor{orange}{#1}}
\newcommand{\green}[1]{\textcolor{deepgreen}{#1}} 
\newcommand{\yellow}[1]{\textcolor{yellow}{#1}}
\newcommand{\white}[1]{\textcolor{white}{#1}}
\newcommand{\grey}[1]{\textcolor{gray}{#1}}


% Folgendes sollte geloescht werden, wenn man nicht am Anfang jedes
% Unterabschnitts die Gliederung nochmal sehen moechte.
% \AtBeginSection[]
% {
% \begin{frame}<beamer>
%   \frametitle{Outline}
%   \tableofcontents[subsectionstyle=show/show/hide,sectionstyle=show/shaded]
%        %   possible styles: show/hide/shaded, or: hideothersubsections
% \end{frame}
% }

% adjust spacing in tableofcontents 
\usepackage{etoolbox}
\makeatletter
\patchcmd{\beamer@sectionintoc}{\vskip1.5em}{\vskip0.5em}{}{}
\makeatother

%   Falls Aufzaehlungen immer schrittweise gezeigt werden sollen, kann
%   folgendes Kommando benutzt werden:

%   \beamerdefaultoverlayspecification{<+->}

%   den Pfad zu den Grafiken einrichten:
\graphicspath{ {bilder/} {../logos/} {./} }

% http://tex.stackexchange.com/questions/84143/fancy-arrows-with-tikz
\tikzfading[name=arrowfading, top color=transparent!0, bottom color=transparent!95]
\tikzset{arrowfill/.style={general shadow={fill=black, shadow yshift=-0.4ex, path fading=arrowfading}}}
\tikzset{smallarrowfill/.style={path fading=arrowfading}}

\tikzset{arrowstyle/.style={draw=blue,arrowfill, top color=blue!20, bottom color=blue, single arrow,minimum height=#1, single arrow, single arrow head extend=1.4ex,}}
\tikzset{smallarrowstyle/.style={draw=blue,smallarrowfill, top color=blue!20, bottom color=blue, single arrow,minimum height=#1, single arrow, single arrow head extend=.6ex,}}
\tikzset{smallredarrowstyle/.style={draw=red,smallarrowfill, top color=red!20, bottom color=red, single arrow,minimum height=#1, single arrow, single arrow head extend=.6ex,}}
\newcommand{\tikzfancyarrow}[2][7ex]{\tikz[baseline=-0.5ex]\node[arrowstyle=#1] {#2};}
\newcommand{\tikzsmallarrow}[2][3ex]{\tikz[baseline=-0.5ex]\node [smallarrowstyle=#1] {#2};}
\newcommand{\tikzsmallredarrow}[2][3ex]{\tikz[baseline=-0.5ex]\node [smallredarrowstyle=#1] {#2};}

% highlight sections of text (no word wrapping)
\usepackage{xspace}
\newcommand\hlwiggly[1]{%
    \tikz[baseline,%
      decoration={random steps,amplitude=1pt,segment length=15pt},%
      outer sep=-15pt, inner sep = 0pt%
    ]%
   \node[decorate,rectangle,fill=gold,anchor=text]{#1\xspace};%
}%

% highlight sections of text (no word wrapping)
\usepackage{xspace}
\newcommand\hl[1]{%
    \tikz[baseline,%
      %decoration={random steps,amplitude=1pt,segment length=15pt},%
      outer sep=-15pt, inner sep = 0pt%
    ]%
   \node[rectangle,fill=gold,anchor=text]{#1\xspace};%
}%


% Draw a shadow underneath an image:
% code adapted from http://tex.stackexchange.com/a/11483/3954
% http://tex.stackexchange.com/questions/81842/creating-a-drop-shadow-with-guassian-blur
% some parameters for customization
\def\shadowshift{3pt,-3pt}
\def\shadowradius{6pt}

\colorlet{innercolor}{black!60}
\colorlet{outercolor}{gray!05}

% this draws a shadow under a rectangle node
\newcommand\drawshadow[1]{
    \begin{pgfonlayer}{shadow}
        \shade[outercolor,inner color=innercolor,outer color=outercolor] ($(#1.south west)+(\shadowshift)+(\shadowradius/2,\shadowradius/2)$) circle (\shadowradius);
        \shade[outercolor,inner color=innercolor,outer color=outercolor] ($(#1.north west)+(\shadowshift)+(\shadowradius/2,-\shadowradius/2)$) circle (\shadowradius);
        \shade[outercolor,inner color=innercolor,outer color=outercolor] ($(#1.south east)+(\shadowshift)+(-\shadowradius/2,\shadowradius/2)$) circle (\shadowradius);
        \shade[outercolor,inner color=innercolor,outer color=outercolor] ($(#1.north east)+(\shadowshift)+(-\shadowradius/2,-\shadowradius/2)$) circle (\shadowradius);
        \shade[top color=innercolor,bottom color=outercolor] ($(#1.south west)+(\shadowshift)+(\shadowradius/2,-\shadowradius/2)$) rectangle ($(#1.south east)+(\shadowshift)+(-\shadowradius/2,\shadowradius/2)$);
        \shade[left color=innercolor,right color=outercolor] ($(#1.south east)+(\shadowshift)+(-\shadowradius/2,\shadowradius/2)$) rectangle ($(#1.north east)+(\shadowshift)+(\shadowradius/2,-\shadowradius/2)$);
        \shade[bottom color=innercolor,top color=outercolor] ($(#1.north west)+(\shadowshift)+(\shadowradius/2,-\shadowradius/2)$) rectangle ($(#1.north east)+(\shadowshift)+(-\shadowradius/2,\shadowradius/2)$);
        \shade[outercolor,right color=innercolor,left color=outercolor] ($(#1.south west)+(\shadowshift)+(-\shadowradius/2,\shadowradius/2)$) rectangle ($(#1.north west)+(\shadowshift)+(\shadowradius/2,-\shadowradius/2)$);
        \filldraw ($(#1.south west)+(\shadowshift)+(\shadowradius/2,\shadowradius/2)$) rectangle ($(#1.north east)+(\shadowshift)-(\shadowradius/2,\shadowradius/2)$);
    \end{pgfonlayer}
}

% create a shadow layer, so that we don't need to worry about overdrawing other things
\pgfdeclarelayer{shadow} 
\pgfsetlayers{shadow,main}

\newsavebox\mybox
\newlength\mylen

\newcommand\shadowimage[2][]{%
\setbox0=\hbox{\includegraphics[#1]{#2}}
\setlength\mylen{\wd0}
\ifnum\mylen<\ht0
\setlength\mylen{\ht0}
\fi
\divide \mylen by 120
\def\shadowshift{\mylen,-\mylen}
\def\shadowradius{\the\dimexpr\mylen+\mylen+\mylen\relax}
\begin{tikzpicture}
\node[anchor=south west,inner sep=0] (image) at (0,0) {\includegraphics[#1]{#2}};
\drawshadow{image}
\end{tikzpicture}}

% set some nice styles to use with tikz
\tikzset{
  terminal/.style={
    % The shape:
    rectangle,minimum size=6mm,rounded corners=3mm,
    % The rest
    very thick,draw=black!50,
    top color=white,bottom color=black!20}
}

\tikzset{fontscale/.style = {font=\relsize{#1}}
    }

