\documentclass[12pt]{article}

\usepackage[utf8x]{inputenc}
\usepackage{verbatim}

\usepackage{url}
\usepackage{hyperref}
\usepackage{float}

\parindent0cm\textwidth15.5cm\oddsidemargin0cm
\textheight23cm\topskip-1cm\headheight0cm\topmargin0cm

\renewcommand{\familydefault}{cmss}
\renewcommand{\seriesdefault}{m}
%\renewcommand{\shapedefault}{n,sc,it,sl}
\renewcommand{\shapedefault}{n}

\begin{document}

\begin{center} {\Large\bf\boldmath KF6 - $\gamma$ Spectroscopy}\\
\end{center}

{\bf Supervisor:}\\
Anton S\r{a}mark-Roth,     2227733, B213, \texttt{Anton.Samark-Roth@nuclear.lu.se}\\
Daniel Cox, 2221707, B220, \texttt{Daniel.Cox@nuclear.lu.se}\\

{\bf The lab:}\\
Starting \underline{8.30 in H212}.
\\

\section*{Learning Outcomes}
\begin{itemize}
  \item Understand what $\gamma$ spectroscopy is.
  \item Know the $\gamma$-ray matter interactions.
  \item Be able to explain the basic principles of radiation detection.
  \item Understand the basic properties of $\gamma$-ray detectors:
  \begin{itemize}
    \item Efficiency
    \item Resolution
  \end{itemize}
  \item Be able to discuss the differences between the two types of detectors; scintillators and semiconductors.
  \item Learn how to measure $\gamma$-ray sources and identify radioactive species.
  \item Be able to interpret $\gamma$-ray spectra and understand how the different features of $\gamma$-ray decays and nuclear structure properties such as $\gamma$-ray matter interactions are highlighted therein.
%\item Measure a series of calibration sources: $^{60}$Co, $^{137}$Cs,  $^{22}$Na, $^{203}$Hg, and $^{232}$Th and exploit their differentfeatures to highlight different aspects in $\gamma$-ray decay.
  \item Be able to measure the deuteron binding energy.
  \item Learn how to perform \textbf{data analysis} in the context of $\gamma$-ray spectroscopy:
  \begin{itemize}
    \item Read in data file.
    \item Perform Gaussian fits to peaks and extract detector resolutions.
    \item Perform detector calibrations.
    \item Calculate full energy peak ratios and internal conversion coefficients.
    \item Perform a statistical analysis of the results, propagate errors and interpret results.
  \end{itemize}
\end{itemize}


\subsection*{Bring}
\begin{itemize}
  %\item Lab manual with all attachments.
  \item Course book.
  \item Laptop.
  \item USB stick (one person per group).
\end{itemize}


\subsection*{Preparations before the lab}
\begin{itemize}
  \item Read Krane 7.1 only on interactions of $\gamma$-rays with matter. %Why is this not included? Too much?
  \item Read Krane 7.3 on scintillator detectors.
  \item Read Krane 7.4 on semiconductor detectors.
  \item Read Krane 7.6 on energy measurements.
  \item Or corresponding chapters in other book.
  \item Complete the preparatory exercise (see next section).
  \item Introduce yourself to data analysis, by walking through the {\it jupyter notebook} and the document on error analysis, named ``error\_analysis.pdf'', located in the zip folder available from the course web page.

\end{itemize}
During the morning you will give a mini-presentation on one of the detectors.

\subsection*{Report}
Reports should be in \texttt{.pdf}. The deadline for handing in the
reports is 10 working days after the laboratory has been conducted.
Please hand in the reports via the course web page (under assignments).
All reports are processed by \textit{Urkund}, a plagiarism checker.


\section*{Exercises before the lab} \label{sec:exe}

{\bf\small Efficiency corrections:}\\
More often than not, one finds that some kind of correction needs to
be applied to experimental data in order to compensate for some
limitation of the instruments. Important quantities for the corrections are presented in Table \ref{tab:I}. In two of the exercises that will be
performed in this lab, we will be interested in the total number of
photons emitted by a radioactive sample during the measurement.  The
number of photons counted by us will deviate from the number $I_0$
emitted by the sample so that we only detect a fraction, $I_{det}$, of
these. The deviation comes about due to three reasons:

\begin{description}
\item[Angular coverage:] The detector does not surround the sample but
    only covers a small part of the solid angle around the sample. For
      isotropic sources it is clear that \emph{only} those $\gamma$ rays
        going into the direction of the detector would be detected.
\item[Efficiency:] All incoming photons do not interact with the
  detector. If we call the number incident on the detector $I_{in}$, only a
  subset, $I$, of these will interact according to the detector
  efficiency, $\epsilon_{eff.}=I_{det}/I_{in}$.
\item[Peak-to-total-ratio:] The measurement we make will result in an
  energy spectrum of the photons that have interacted with the
  detector. A photon will deposit some or all of its energy in the
  detector depending on the way in which it interacts (see Krane ch. 7.1). We will count the ones that deposit all of its energy in
  the detector (these are fully absorbed and denoted full energy), which means that we get a subset,
  $I_{fe}$, of the number of photons, $I_{in}$, that interact with the
  detector. The number of fully absorbed photons $I_{fe}$ (full energy, the ones we
  count) is related to $I_{det}$ through $\epsilon_{P/T} = I_{fe}/I_{det}$, where
  $P/T$ is called the peak-to-total-ratio. Since we are interested in
  knowing the total number that interacted with the detector, we need
  to correct for this as well.
\end{description}

\begin{table}[H]
  \centering
  \caption{Quantities important for efficiency corrections.}
\begin{tabular}{c l}
  \hline
  $I_0$ & $\gamma$-rays emitted by the source \\
  $I_{in}$ & $\gamma$-rays incident on the detector \\
  $I_{det}$ & $\gamma$-rays detected \\
  $I_{fe}$ & $\gamma$-rays detected with full energy \\
  \hline
  $\epsilon_{eff.}$ & Intrinsic efficiency of detector \\
  $\epsilon_{P/T}$ & Peak-to-total-ratio \\
\end{tabular}
\label{tab:I}
\end{table}


{\bf\small Task:}\\
In preparation for the lab, write down an expression to correct for
these effects so that we get an estimate of the total number of
photons, $I_0$, emitted by the sample. Assume that we know the solid
angle $\Omega$ of the detector with respect to the sample, and that
the sample is emitting isotropically (all emission angles are equally
probable). Also assume that we know the values of $\epsilon_{P/T}$ and
that it corresponds to the energy of the radiation we are looking
at. During the laboratory we will see that the correction factors
$\epsilon_{P/T}$ and $\epsilon_{eff.}$ depend on the energy of the
incoming radiation but assume here that you are looking at incoming
radiation with constant energy. Thus, the correction factors are
constants.\\

{\bf\small Yield corrections:}\\
Certain nuclear deexcitations do not always occur with the emission of
a gamma ray. Sometimes a phenomenon called {\it internal conversion}
happens. Before the lab get acquainted with the phenomena and discuss
how the efficiency correction is affected by this.\\

{\bf\small Gamma-ray detectors:}\\
We will use two common detector types during the lab: a scintillator
and a semiconductor detector. During the lab, you will be divided into
two groups and each group will be assigned one of the detector types
and asked to explain it to the other group. In preparation for this,
read about the detectors in the indicated pages in
Krane so that you can explain the main principles in simple terms. We
will use a NaI(Tl)-detector, which is a crystal scintillator and a
High Purity Germanium detector (HPGe) which is a semiconductor. It is
enough to describe in general terms how the semiconductor works, do
not focus too much on the specifics of the HPGe-detector. The same goes for the scintillator and the NaI(Tl) detector.
\end{document}
