\documentclass[a4,11pt, notitlepage]{report}

\usepackage{fixltx2e}

\usepackage[utf8]{inputenc}
\usepackage[T1]{fontenc}
\usepackage[english]{babel} 
\usepackage{graphicx}
\usepackage{textcomp}
\usepackage{mathpazo}
\usepackage{fullpage}
\usepackage{epsfig}
\usepackage{subfigure}
\usepackage{listings}
\usepackage{SIunits}
\usepackage{rotating}
\usepackage[hmargin=3cm,vmargin=3.5cm]{geometry}  

%\usepackage{helvet}

%https://twiki.cern.ch/twiki/bin/view/ALICE/PWG4HighPtDeDx


\begin{document} 


\begin{center}
\Large{Learning outcomes for the lab exercise KF7: $\beta$-spectrum and Fermi-Kurie plot}
\end{center}


Laborationens mål är, i enlighet med kursplanen för FYSC12: Kärnfysik och Reaktorer, 7.5 hp, att studenten efter genomgången laboration skall:

\begin{itemize}
\item känna till grundläggande egenskaper hos beta-sönderfall;
\item kunna beskriva huvuddragen i omskrivningen av Fermis
  Gyllende regel, inklusive de approximationer som gjorts;
\item förstå hur -- och varför -- ett Fermi-Kurie-histogram
  görs för att bestämma Q-värdet för ett beta-sönderfall;
\item känna till huvudprinciperna för hur en plastscintillator-detektor fungerar;
 \item kunna diskutera lämplig uppställning och tillvägagångssätt för
   experimentet;
\item kunna -- med hjälp av handledaren -- hantera ett radioaktivt preparat
  och använda en plastscintillator (inklusive kalibrering) för mätning av sönderfallspartiklars energi;
\item värdera experimentella resultat.

\end{itemize}

In accordance with the syllabus of FYSC12: Nuclear Physics and Reactors, 7.5
credits, the goal of the laboration exercise is that the student -- after
completed the lab -- shall:

\begin{itemize}
\item be familiar with the basic properties of beta-decay;
\item be able to describe the main features of the rewriting of the
  Fermi Golden rule, including approximations made;
\item understand how -- and why -- a Fermi Kurie plot is done in the purpose
  of determine the Q value for a beta-decay;
\item know the main principles of how a plastic scintillator detector works;
\item be able to discuss the proper setup and procedures of the experiment;
\item with the help of the supervisor handle radioactive samples and use a
  plastic scintillator (including calibration) to measure the decay particle energy;
\item evaluate experimental results.
\end{itemize}


\end{document}