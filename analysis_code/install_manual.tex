%%% TeX-command-extra-options: "-shell-escape"
% IMPORTANT: compiling from the command line using:
% pdflatex -shell-escape jupyter.tex

\documentclass[a4,11pt, notitlepage]{article}

\usepackage{fixltx2e}
\usepackage[utf8]{inputenc}
\usepackage[T1]{fontenc}
\usepackage[english]{babel}
\usepackage{graphicx}
\usepackage{textcomp}
\usepackage{mathpazo}
\usepackage{fullpage}
\usepackage{epsfig}
\usepackage{color}
\usepackage{subfigure}
\usepackage{listings}
\usepackage{SIunits}
\usepackage{rotating}
\usepackage{url}
\usepackage[hmargin=3cm,vmargin=3.5cm]{geometry}
\usepackage{listings}

\begin{document}


\title{Instructions for FYSC12 labs:\\ How to install and run Jupyter Notebooks}
%\author{}
\date{}
\maketitle

\vspace{-1cm}
%\section{Data Analysis}
%\label{sec:data-analysis}

The ``online'' visualization that the Maestro MCA software allows is
very helpful but limited and tedious to use for multiple data samples
or to estimate systematic uncertainties
-- no way around writing our own code!

Lucky for you, key aspects of the analysis are already coded and
uploaded to Live@Lund. The next
sections explain what is needed in order to run the analysis and how
to do so.

\section{Installing Prerequisites}
\label{sec:prerequisites}

The analysis is written in Python3 with the SciPy
library\footnote{\url{http://www.scipy.org}} for scientific computing and run
inside a Jupyter Notebook\footnote{\url{https://jupyter.org/}} -- it's all
  free and open software, available for all major platforms and
  highly flexible; and the key packages are just a download away:
  Anaconda\footnote{\url{https://www.continuum.io/downloads}, choose the version
    for Python 3!} provides the complete toolkit we need.
  You can find more detailed installation instructions online.\footnote{For
  example: \url{https://jupyter.readthedocs.io/en/latest/install.html}}


\section{Running the Analysis}
\label{sec:running-analysis}
Download the zipped code from Live@Lund into a directory
of your liking and extract the file contents.

\subsection*{Conda environment}
An environment in conda\footnote{\url{https://conda.io/docs/user-guide/tasks/manage-environments.html}}, makes it easy to have portable code, i.e. program that can run on any computer with any operative system.
The {\emph FYSC12} environment is described by the file {\emph environment.yml} in the folder 'analysis\_code'.
You have two options on how to install and activate the environment:

\newpage
\subsection*{1 - With the Anaconda Navigator (recommended)}
\begin{itemize}
  \item Open the navigator.
  \item Click 'Environment' on the left-hand menu.
  \item Click 'Import' on the bottom of the GUI.
  \item Navigate to the folder you downloaded and choose the file 'environment.yml'.
  \item Once the installation is completed, click on the 'play' button to activate the environment.
  \item NOTE: whenever you want to run the FYSC12 notebooks you need to activate the environment first.
\end{itemize}

\subsection*{2 - From a command prompt}

Run the following in a command prompt (also called \emph{Terminal} or \emph{Konsole} (also possible with the conda prompt):

\begin{lstlisting}{bash}
cd path/to/the/directory/you/chose
conda env create -f environment.yml
(On Linux) source activate FYSC12-Labs
(On Mac) source activate FYSC12-Labs ?
(With conda prompt) conda activate FYSC12-Labs ?
(On Windows) conda activate FYSC12-Labs ?
./postBuild
\end{lstlisting}


\subsection*{Open the jupyter notebook}
Now start up the Jupyter server by \\


Look for and click the \emph{Jupyter Notebook} icon in the Anaconda navigator. \\


Alternatively:
\begin{lstlisting}{bash}
cd path/to/the/directory/you/chose
jupyter notebook
\end{lstlisting}
on the command prompt (also called \emph{Terminal} or \emph{Konsole}
depending on the OS). \\

Now your default web browser should launch and display the Jupyter application
running at \url{http://localhost:8888}. Navigate to the Jupyter notebook you
want to use for the lab and select it to open it in a new tab.

\section*{For the KF7-Beta lab}
In the opened jupyter server navigate to the folder KF7-Beta and open the file 'beta-lab\_notebook.ipynb'.
Follow the instructions in the notebook at least until the point where you see the first plot -- and you are all set for the lab!


\end{document}


