\documentclass[11pt]{article}

\usepackage[utf8x]{inputenc}
\usepackage{verbatim}

\usepackage{url}

\parindent0cm\textwidth15.5cm\oddsidemargin0cm
\textheight23cm\topskip-1cm\headheight0cm\topmargin0cm

\renewcommand{\familydefault}{cmss}
\renewcommand{\seriesdefault}{m}
%\renewcommand{\shapedefault}{n,sc,it,sl}
\renewcommand{\shapedefault}{n}

\usepackage{fancyhdr}
\pagestyle{fancy}
\rfoot{\tiny 09/2019, ASR}
\cfoot{}

\begin{document}

\begin{center}

  {\Large\bf\boldmath Risk assessment for FYSC12\\ Nuclear Physics
    laboratories:}\\[1mm] {\large\bf\boldmath \textsc{KF1}, \textsc{KF3},
    \textsc{KF6}, and \textsc{KF7}}\\
  \vspace*{4mm}

  {\footnotesize Eric Corrigan, Joochun Park, Hanno Perrey, Emil
    Rofors, Anton S{\aa}mark-Roth, Daniel Cox, Nicholai Mauritzson, and Luis
    Sarmiento}\\
  {\small\it Department of Physics, Lund University,
    S-22100 Lund, Sweden}\\

\end{center}

{\bf Risks:}
\begin{itemize}
\item Ionizing radiation from alpha, beta, neutron, and gamma sources.
\item Liquid Nitrogen frostbite
\item Electrical shock
\item Physical harm with heavy shielding material (lead).
\item Toxic liquids in the vicinity.
\end{itemize}

{\bf General Rules for the Laboratories:}
\begin{itemize}
\item This document must be read and signed by all parties involved in
  the nuclear physics laboratories.
\item It is forbidden to eat, drink, smoke, snuff (snus) and apply
  cosmetics in the lab.
\item Heavy equipment must be transported by at least two people.
\item Lead shielding may only be moved, if gloves and safety shoes are
  used.
\item Do not block any escape routes, take fire safety into
  consideration. Check where the next fire extinguisher is located and
  know the location of the nearest emergency exits and assembly
  points.
\item This is a living document. It will be checked and updated
  regularly.
\end{itemize}

{\bf Radioactive Sources:}
\begin{itemize}
\item The University’s general safety instructions regarding ionizing
  radiation must be read and understood by all parties involved.
  Available in zip folder from course webpage.
  %(\path{http://www.staff.lu.se/sites/staff.lu.se/files/lund-university-radiation- protection-regulations.pdf})
\item People working at the lab are required to bring their own
  radiation badge. Accompanying persons should be in company of a lab
  worker wearing a radiation badge.
\item Prior to using a radioactive source, all parties must thoroughly
  discuss the task at hand so that they understand: – the radiological
  situation, especially the dose rate;– their role in the use of the
  source i.e. handler or observer;– where they should position
  themselves during the use of the source; – what they should be doing
  at all times during the use of the source.
\item Both the source handler and observers must apply standard ALARA
  principles during the usage of the source.
\item Prior to use, a walkthrough of the task at hand without the
  source shall be performed as an ALARA rehearsal.
\item Always move slowly and methodically. Do not rush your activity.
\item Wash your hands thoroughly with soap and water after using
  radioactive sources.
\item In the laboratories mixed-field gamma/neutron source may be
  used.  Special care has to be taken to reduce the exposer of all
  parties to the neutron field.
\item Personal radiation monitors should be used to evaluate new
  (unknown) sources, this is especially important when using neutron
  sources.
\item In case of use of thin window beta and gamma sources they have
  to be handled with additional care by all parties (Do not damage
  the window.)
\end{itemize}

{\bf Liquid Nitrogen:}
\begin{itemize}
\item Liquid nitrogen (LN$_2$) may cause frostbite by contact with
  skin or eyes. Gloves, safety glasses and closed shoes have to be
  worn whenever LN$_2$ is handled.
\item Liquid nitrogen may only be used, transported etc. after
  instruction by experienced personal.
\item Vessels containing LN$_2$ must be labeled.
\item Liquid nitrogen must not be stored in sealed containers. There
  is a risk of explosion.
\item Ensure sufficient ventilation when storing and handling
  LN$_2$. Evaporating LN$_2$ might cause suffocation.
\item When LN$_2$ is to be transported using a lift two persons should
  be involved at origin and destination levels. Accompanying persons
  must not be in the lift when LN$_2$ is transported.
\item LN$_2$ vessel must be secured against spilling and tripping when
  transported and stored.
\end{itemize}

{\bf High Voltage and Electronics:}
\begin{itemize}
\item Make sure the equipment is properly earthed.
\item Whenever possible use ground fault circuit breakers.
\item Switch off the power when handling electrical connections.
\item Check cables for damage before use.
\item In case of electrical fire use CO$_2$ fire extinguishers.
\end{itemize}

\vspace{15mm}
{\Large Name:}
\begin{center}
\vspace{-5mm}
\rule{\textwidth}{1pt}%
\vspace{5mm}
Signature / Date: \rule{10cm}{0.5pt}%

\end{center}

\end{document}
