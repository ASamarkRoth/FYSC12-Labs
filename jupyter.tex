%%% TeX-command-extra-options: "-shell-escape"
% IMPORTANT: compiling from the command line using:
% pdflatex -shell-escape jupyter.tex

\documentclass[a4,11pt, notitlepage]{article}

\usepackage{fixltx2e}
\usepackage[utf8]{inputenc}
\usepackage[T1]{fontenc}
\usepackage[english]{babel} 
\usepackage{graphicx}
\usepackage{textcomp}
\usepackage{mathpazo}
\usepackage{fullpage}
\usepackage{epsfig}
\usepackage{color}
\usepackage{subfigure}
\usepackage{listings}
\usepackage{SIunits}
\usepackage{rotating}
\usepackage{url}
\usepackage[hmargin=3cm,vmargin=3.5cm]{geometry}  
\usepackage{listings}

\usepackage{minted}

\begin{document} 
 

\title{Instructions for FYSC12 labs:\\ How to install and run Jupyter Notebooks}
%\author{}
\date{}
\maketitle

\vspace{-1cm}
%\section{Data Analysis}
%\label{sec:data-analysis}

The ``online'' visualization that the Maestro MCA software allows is
very helpful but limited and tedious to use for multiple data samples
or to estimate systematic uncertainties
-- no way around writing our own code!

Lucky for you, key aspects of the analysis are already coded and
uploaded to Live@Lund. The next
sections explain what is needed in order to run the analysis and how
to do so.

\section{Installing Prerequisites}
\label{sec:prerequisites}

The analysis is written in Python3 with the SciPy
library\footnote{\url{http://www.scipy.org}} for scientific computing and run
inside a Jupyter Notebook\footnote{\url{https://jupyter.org/}} -- it's all
  free and open software, available for all major platforms and
  highly flexible; and the key packages are just a download away:
  Anaconda\footnote{\url{https://www.continuum.io/downloads}, choose the version
    for Python 3!} provides the complete toolkit we need.
  You can find more detailed installation instructions online.\footnote{For
  example: \url{https://jupyter.readthedocs.io/en/latest/install.html}}

\section{Running the Analysis}
\label{sec:running-analysis}

Download the zipped code from Live@Lund into a directory
of your liking and extract the file contents.

Now start up the Jupyter server by running
\begin{minted}{bash}
cd path/to/the/directory/you/chose
jupyter notebook
\end{minted}
on the command prompt (also called \emph{Terminal} or \emph{Konsole}
depending on the OS). Alternatively, you can look for and click the \emph{Jupyter Notebook}
icon installed by Anaconda.

Now your default web browser should launch and display the Jupyter application
running at \url{http://localhost:8888}. Navigate to the Jupyter notebook you
want to use for the lab and select it to open it in a new tab.

Follow the instructions in the notebook at least until the point where you see the first
plot -- and you are all set for the lab!


\end{document}


